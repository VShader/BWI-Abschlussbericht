\documentclass[12pt]{article}
\usepackage[utf8]{inputenc}
\usepackage[german]{babel}
\usepackage{textcomp}
\usepackage[T1]{fontenc}

\title{BWI Abschlussbericht}
\author{Kortemeier, Sascha, Matr.-Nr. 1 \\
Weitz, Tobias , Matr.-Nr. 2 \\
Hoetz, Marcel, Matr.-Nr. 3}
\date{January 2015}

\begin{document}

\maketitle

\section{Strategie}

Welches Ziel haben Sie sich für die letzte Periode gesetzt? 
Welche Basisstrategie wurde verfolgt, um dieses Ziel zu erreichen?
Mit welchen konkreten Entscheidungen wurde diese Strategie umgesetzt?
\\
\\
Unsere Strategie bestand darin einen großen Absatz zu erreichen. 
Um das Ziel zu erreichen haben wir in der ersten Periode Maschinen und Personal eingestellt. 
Um die produzierte Ware verkaufen zu können haben wir einen konstant niedrigen Preis gewählt und diesen dann an die jeweiligen Perioden anzupassen. 
Dazu gehört der Vergleich mit Preisen anderer Unternehmen und das Marktwachstum. 

\subsection{Periode 1:}
Periode 1 war eine Leerlauf Periode mit Standardwerten für alle Unternehmen.	

\subsection{Periode 2:}
In der zweiten Periode haben wir die Grundlage für eine hohe Produktion geschaffen.
Dazu haben wir zwei Fertigungsanlagen angeschafft und 4 Mitarbeiter in der Produktion eingestellt.
Um den Kreditrahmen nicht zu überreizen haben wir uns dazu entschlossen die Ausgaben für Forschung \& Entwicklung und Marketing gering zu halten.
Somit hatten wir für die Produktion 4 Fertigungsanlagen und 9 Mitarbeiter für die Produktion.
Dadurch waren wir in der Lage 2000 Zelte zu produzieren.
Um diese Menge absetzen zu können, haben wir den Verkaufspreis gering gehalten.
Weitere Gründe für den geringen Verkaufspreis waren zum einen der geringe Forschungsindex und zum anderen die geringen Ausgaben für Werbung.

\subsection{Periode 3:}
Da in dieser Periode der Mengenrabatt hoch war haben wir zusätzlich 400 Werkstoffe eingekauft. 

\subsection{Periode 4:}
Da der Markt um 10\% - 11\% wachsen soll und aus den prof. Absatz aus Periode 3 von 1683 Zelten sind wir davon Ausgegangen 1900 Zelte zu verkaufen.
Um dies sicherzustellen haben wir zusätzlich einen Mitarbeiter im Vertrieb eingestellt.
Da in dieser Periode ein Mitarbeiter gekündigt hat und der Rest 5\% weniger effektiv arbeitet können wir  1710 Zelte produzieren.
Zudem hatten wir 464 Zelte auf Lager (in der Summe 2174 Zelte).
Mitarbeiter einzustellen war in dieser Periode teurer als sonst,
deshalb haben wir an dazu entschlossen die Produktion von 1700 zu halten und die Lager zu leeren.
F\& E wurde auf 100000\texteuro aufgestockt um den Anschluss nicht zu verlieren.  

\subsection{Periode 5:}
Für Periode 5 haben wir uns entschieden Material für 2 Perioden zu kaufen.
Dadurch konnten wir die Herstellungskosten für diese Periode um 13\texteuro pro Zelt und für die nächste Periode um 8\texteuro pro Zelt senken.

\subsection{Periode 6:}
Aus der letzten Periode haben wir erkannt, dass es nicht möglich ist zweitausend Zelte zu einem Preis von 480\texteuro zu verkaufen.
Deshalb haben wir uns dazu entschieden den Preis wieder zu senken.
Da der Markt in dieser Periode laut Prognose um 10\% schrumpfen soll haben wir den Preis von 480\texteuro auf 440\texteuro gesenkt und die Gelder für die Verkaufsfördernd erhöht.
Um unserer Kosten zu senken haben wir die Gelder für Printwerbung und Forschung \& Entwicklung gesenkt, da die Auswirkungen auf deren Verkauf aus unserer Sicht nicht mehr so groß waren.


\subsection{Periode 7:}
Da wir unser Lagerbestand hoch war, wollten wir mit einem niedrigen Preis dafür sorgen,
dass der Lagerbestand gegen null geht.
Da zwei andere Unternehmen ebenfalls einen hohen Absatz erzielen wollten, habe wir uns für einen sehr niedrigen Preis entschieden.


\section{Daten und Kennzahlen}
Folgende Kennzahlen sollen zur Unternehmenszustandsbeschreibung benutzt werden:
den Periodenüberschuss (+) bzw. den Periodenfehlbetrag
den kumulierten Periodenüberschuss
den kumulierten Erfolgswert
die Herstellkosten
den Lagerbestand an Fertigprodukten
die Auslastung der Fertigung / des Personals
den Quotient aus Überziehungskredit / Bilanzsumme (-) 
bzw. aus Kassenbestand / Bilanzsumme (+)
Führen Sie bitte grafisch und in Tabellenform den Verlauf dieser Kennzahlen für alle Planspielperioden (ohne den Probelauf) an, und zwar
die jeweiligen Kennzahlen Ihres Unternehmen
die jeweils beste erreichte Kennzahl in Ihrem Markt
die jeweils schlechteste erreichte Kennzahl in Ihrem Markt



\section{Kritische Würdigung der Strategie}
Beurteilen Sie, inwieweit Sie Ihre Strategie umsetzen konnten bzw. eingehalten haben.
Bitte beziehen Sie sich dabei auf die unter 2. aufgeführten Kennzahlenverläufe.
Wenn zur Erläuterung weitere Kennzahlen notwendig sind, so führen Sie diese bitte hier auf.

Durch die frühe Anschaffung von Maschinen und Personal sowie des niedrigen Preises war es uns schon früh möglich einen großen Absatz zu erzielen.
Unseren Preis haben wir immer an den aktuellen Markt sowie den Preisen der Konkurrenz angepasst.
Leider haben wir einmal bei schwacher Konjunktur nicht genug Ware an den Markt bringen können wodurch wir unsere Überkapazität an den Großhändler zu einem viel niedrigerem Preis abtreten mussten.
Glücklicherweise konnten wir von der zu niedrigen Produktionskapazität verschiedener Unternehmen profitieren.
Leider haben wir es zum ende nicht geschaft den größten Absatzt zu erzielen und andere Unternehmen aus dem Markt zu drängen.
Dafür aber zu dem zweitgrößten Absatzt bei einem unterdurchschnittlichem niedrigen Preis mit durchschnittlicher Qualität.

\section{Die Situationsdarstellung des Unternehmens nach Periode 8 (?)}  
Beschreiben Sie die aktuelle Situation Ihres Unternehmens im Vergleich zu Ihren Wettbewerbern. Wo liegen Ihre Stärken, wo liegen Ihre Schwächen? Belegen Sie Ihre Aussagen mit den zugehörigen Kennzahlen.
Im Nachhinein war unserer Preis zu niedrigen im Vergleich zur menge der Produzierbaren Zelte.
Unser Potentieller Absatz von Zelten lag bei ca. 2500, wohingegen die unserer maximale Produktionskapazität bei 2000 lag.
Durch diese große Differenz konnten vor allem die anderen Unternehmen profitieren.
Für die nächste Periode würden wir unser Unternehmen um eine Anlage und erweitern und zusätzliches Personal in der Produktion einstellen.
Damit wären wir in der Lage 2500 Zelte zu produzieren was unserem aktuellem potentiellen Absatz entspricht.
Zudem hätten wir noch den Vorteile das wir den Mengenrabatt von Nylon und Gestänge besser ausnützen könnten.
5.	Ausblick auf Periode 12  (?)
Beschreiben Sie bitte, inwieweit Sie die unter 1. beschriebene Strategie weiter führen oder abändern würden, wenn Sie noch vier weitere Perioden spielen müssen.
Wir würden versuchen unseren Absatz von 2500 zu Zelten halten und evtl. auszubauen.
Einer unserer Fehler, den Großhändler mit in die Produktionsplanung mit einzubeziehen, würden wir nicht mehr wiederholen.
Stattdessen würden wir ihn nur noch in schlechten Perioden nutzen um eine gesicherte
Einnahmequelle zu erhalten und unseren Produktionsüberfluss zu kompensieren.


\section{Erfahrungen in und aus der Teamarbeit}
Beschreiben Sie bitte Ihre Erfahrungen aus der Teamarbeit:
Was war gut, was schlecht?
Was hat die Gruppe gelernt?
Gab es organisatorische Probleme?
Gab es Arbeitsteilung? Wie war sie organisiert?
Was würden Sie beim nächsten Mal anders machen?
Welche Verbesserungsvorschläge haben Sie an die Planspielleitung?



Da unsere Gruppe nur aus 3 Mitgliedern bestand war es einfach Entscheidungen zu treffen.
Da wir uns schnell auf ein Ziel einigen konnten, waren sich alle aus der Gruppe schnell einig.
Ein Nachteil bei unserer Organisation war das nur eine Person aus der Gruppe zugriff auf die Dropbox hatte.
Dadurch, dass derjenige mit dem Zugang immer da sein musste.
Diese Bedrohung einen Ausfalles hat sich im Nachhinein jedoch als positiv erwiesen denn er hat zu einer höheren Disziplin geführt, sodass alle Gruppenmitglieder sich intensiv mit der Planung beschäftigt haben.
Eine strikte Aufgabenteilung gab es während des Planspiels nicht.
Zusammenfassend kann man sagen dass die Gruppenarbeit sehr gut lief.
Einziger Vierbesserungspunkt wäre  der Zugang zur Dropbox. 
Anhang



Die Pflichtinhalte ergeben sich aus den nachfolgenden Gliederungspunkten.
Prognoserechnung
Absatz- und Produktionsplanung
Kostenrechnung
Finanzplanung
Ergebnisplanung (GuV)





\end{document}
